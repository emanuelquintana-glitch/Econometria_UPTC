\documentclass[6pt, landscape]{extarticle}

% ========================================
% PAQUETES ESENCIALES
% ========================================
\usepackage[landscape, margin=0.3in, top=0.4in, bottom=0.3in]{geometry}
\usepackage{multicol}
\usepackage{xcolor}
\usepackage{titlesec}
\usepackage{amsmath, amssymb, amsthm, mathtools}
\usepackage{microtype}
\usepackage{booktabs}
\usepackage{enumitem}
\usepackage{tikz}
\usepackage{tcolorbox}
\usepackage{array}
\usepackage{tabularx}
\usepackage{graphicx}

% ========================================
% TIPOGRAFÍA
% ========================================
\usepackage{tgtermes} 
\usepackage[scaled=0.85]{helvet}

% ========================================
% COLORES UPTC
% ========================================
\definecolor{DeepBlack}{HTML}{1A1A1A}
\definecolor{GoldLeaf}{HTML}{A68966}
\definecolor{MITRed}{HTML}{A31F34}
\definecolor{UPTCAzul}{HTML}{003366}
\definecolor{BoxBg}{HTML}{F8F9FA}
\definecolor{FormulaBox}{HTML}{FFF8E7}
\definecolor{WarningBox}{HTML}{FFF0F0}

% ========================================
% ESTÉTICA DE SECCIONES
% ========================================
\titleformat{\section}
  {\color{UPTCAzul}\normalfont\small\bfseries\sffamily\centering}
  {}{0em}{}
  
\newcommand{\sectionline}{%
  \nopagebreak\vspace{-4pt}\noindent\textcolor{GoldLeaf}{\rule{\linewidth}{1pt}}\par\vspace{2pt}%
}

\titleformat{\subsection}
  {\color{MITRed}\normalfont\scriptsize\bfseries\sffamily}
  {$\blacktriangleright$}{0.3em}{}

% ========================================
% CAJAS PERSONALIZADAS
% ========================================
\newtcolorbox{formulabox}{
  colback=FormulaBox,
  colframe=GoldLeaf,
  boxrule=0.5pt,
  arc=2pt,
  left=2pt,
  right=2pt,
  top=1pt,
  bottom=1pt
}

\newtcolorbox{warningbox}{
  colback=WarningBox,
  colframe=MITRed,
  boxrule=0.5pt,
  arc=2pt,
  left=2pt,
  right=2pt,
  top=1pt,
  bottom=1pt
}

% ========================================
% AJUSTES
% ========================================
\setlist[itemize]{leftmargin=*, noitemsep, topsep=0pt, parsep=0pt}
\setlist[enumerate]{leftmargin=*, noitemsep, topsep=0pt, parsep=0pt}
\setlength{\columnseprule}{0.3pt}
\def\columnseprulecolor{\color{black!15}}
\setlength{\columnsep}{1.2em}
\setlength{\parindent}{0pt}
\setlength{\parskip}{1pt}

% ========================================
% COMANDOS MATEMÁTICOS
% ========================================
\newcommand{\E}{\mathbb{E}}
\newcommand{\Var}{\text{Var}}
\newcommand{\Cov}{\text{Cov}}
\newcommand{\FIV}{\text{FIV}}
\newcommand{\TOL}{\text{TOL}}
\newcommand{\Xmat}{\mathbf{X}}
\newcommand{\Ymat}{\mathbf{Y}}
\newcommand{\betahat}{\hat{\boldsymbol{\beta}}}

% ========================================
% INICIO DEL DOCUMENTO
% ========================================
\begin{document}

\begin{center}
    {\sffamily\fontsize{6}{7}\selectfont \color{gray} 
    DEPARTAMENTO DE ECONOMÍA \textbullet\ UPTC TUNJA \textbullet\ ECONOMETRÍA AVANZADA} \\[0.3em]
    {\rmfamily\itshape\fontsize{22}{22}\selectfont \color{DeepBlack} 
    Multicolinealidad: Teoría, Diagnóstico y Remedios} \\[0.2em]
    {\color{GoldLeaf}\rule{0.5\textwidth}{1.2pt}} \\[0.3em]
    {\sffamily\bfseries\fontsize{6}{7}\selectfont 
    VOL. III --- FEBRERO 2026 --- PREPARADO POR EMANUEL}
\end{center}

\vspace{-8pt}

\begin{multicols*}{4}
\raggedcolumns

% ========================================
% COLUMNA 1
% ========================================

\section{I. FUNDAMENTOS TEÓRICOS}
\sectionline

\subsection{DEFINICIÓN FORMAL}

La multicolinealidad ocurre cuando:
\begin{formulabox}
\vspace{-3pt}
$$\exists \; \lambda_1, \ldots, \lambda_k \in \mathbb{R} : \sum_{j=1}^{k} \lambda_j \Xmat_j \approx \mathbf{0}$$
\vspace{-3pt}
\end{formulabox}

\textbf{Consecuencia directa:}
\begin{itemize}
    \item $\det(\Xmat'\Xmat) \approx 0$
    \item $(\Xmat'\Xmat)^{-1}$ tiene elementos grandes
    \item $\Var(\betahat) \to \infty$
\end{itemize}

\subsection{TIPOS DE MULTICOLINEALIDAD}

\textbf{1. Perfecta:} $|\Xmat'\Xmat| = 0$
\begin{itemize}
    \item Estimadores indeterminados
    \item $\text{se}(\betahat) = \infty$
    \item Imposible calcular MCO
\end{itemize}

\textbf{2. Imperfecta:} $|\Xmat'\Xmat| \approx 0$
\begin{itemize}
    \item Estimadores calculables
    \item Varianzas muy grandes
    \item Alta inestabilidad
\end{itemize}

\subsection{CAUSAS COMUNES}

\begin{enumerate}
    \item \textbf{Series de tiempo}: Variables con tendencias comunes
    \item \textbf{Micronumerosidad}: $n$ pequeño vs. $k$ grande
    \item \textbf{Términos polinomiales}: $X, X^2, X^3$ correlacionados
    \item \textbf{Especificación}: Variables que miden lo mismo
    \item \textbf{Muestreo}: Rango limitado de regresores
\end{enumerate}

\subsection{MATRIZ DE INFORMACIÓN}

\begin{formulabox}
\vspace{-3pt}
$$\Xmat'\Xmat = \begin{bmatrix}
n & \sum X_2 & \sum X_3 \\
\sum X_2 & \sum X_2^2 & \sum X_2X_3 \\
\sum X_3 & \sum X_2X_3 & \sum X_3^2
\end{bmatrix}$$
\vspace{-3pt}
\end{formulabox}

\textbf{Propiedades clave:}
\begin{itemize}
    \item Simétrica: $(\Xmat'\Xmat)' = \Xmat'\Xmat$
    \item Dimensión: $(k \times k)$
    \item Definida positiva si no hay colinealidad
\end{itemize}

\subsection{ESTIMADOR MCO}

\begin{formulabox}
\vspace{-3pt}
$$\betahat = (\Xmat'\Xmat)^{-1}\Xmat'\Ymat$$
\vspace{-3pt}
\end{formulabox}

\textbf{Condición de existencia:}
$$(\Xmat'\Xmat)^{-1} \text{ existe } \iff \text{rango}(\Xmat) = k$$

% ========================================
% COLUMNA 2
% ========================================

\section{II. MATRIZ VAR-COV Y FIV}
\sectionline

\subsection{VARIANZA DE LOS ESTIMADORES}

\begin{formulabox}
\vspace{-3pt}
$$\Var(\betahat) = \sigma^2(\Xmat'\Xmat)^{-1}$$
\vspace{-3pt}
\end{formulabox}

\textbf{Expansión para 3 variables:}
\begin{formulabox}
\vspace{-3pt}
{\scriptsize
$$
\begin{aligned}
&\sigma^2(\Xmat'\Xmat)^{-1} = \\
&\begin{bmatrix}
\Var(\hat{\beta}_1) & \Cov(\hat{\beta}_1,\hat{\beta}_2) & \Cov(\hat{\beta}_1,\hat{\beta}_3) \\
\Cov(\hat{\beta}_2,\hat{\beta}_1) & \Var(\hat{\beta}_2) & \Cov(\hat{\beta}_2,\hat{\beta}_3) \\
\Cov(\hat{\beta}_3,\hat{\beta}_1) & \Cov(\hat{\beta}_3,\hat{\beta}_2) & \Var(\hat{\beta}_3)
\end{bmatrix}
\end{aligned}
$$
}
\vspace{-3pt}
\end{formulabox}


\subsection{EXPLOSIÓN DE VARIANZA}

\begin{warningbox}
\textbf{Fórmula crítica:}
\vspace{-3pt}
$$\Var(\hat{\beta}_j) = \frac{\sigma^2}{\sum x_j^2} \cdot \underbrace{\frac{1}{1-R_j^2}}_{\FIV_j}$$
\vspace{-3pt}
\end{warningbox}

Donde:
\begin{itemize}
    \item $R_j^2$ = $R^2$ de regresión auxiliar de $X_j$ sobre demás
    \item $\FIV_j$ = Factor de Inflación de Varianza
\end{itemize}

\subsection{TABLA FIV vs. $R_j^2$}

\begin{center}
{\tiny
\begin{tabular}{@{}ccc@{}}
\toprule
$R_j^2$ & $\FIV_j$ & Inflación \\
\midrule
0.00 & 1.0 & Ninguna \\
0.50 & 2.0 & $\times 2$ \\
0.80 & 5.0 & $\times 5$ \\
\rowcolor{WarningBox}
0.90 & 10.0 & \textbf{CRÍTICO} \\
0.95 & 20.0 & $\times 20$ \\
\rowcolor{WarningBox}
0.99 & 100.0 & \textbf{CATASTRÓFICO} \\
1.00 & $\infty$ & Perfecta \\
\bottomrule
\end{tabular}
}
\end{center}

\subsection{FACTOR DE TOLERANCIA}

\begin{formulabox}
\vspace{-3pt}
$$\TOL_j = \frac{1}{\FIV_j} = 1 - R_j^2$$
\vspace{-3pt}
\end{formulabox}

\textbf{Interpretación:}
\begin{itemize}
    \item $\TOL \to 1$: Independencia total
    \item $\TOL < 0.10$: \textcolor{red}{\textbf{Problema serio}}
    \item $\TOL \to 0$: Colinealidad perfecta
\end{itemize}

\subsection{EFECTO EN PRUEBAS $t$}

Si $R_j^2 \to 1$:
\begin{formulabox}
\vspace{-3pt}
$$\text{se}(\hat{\beta}_j) \to \infty \implies t = \frac{\hat{\beta}_j}{\text{se}(\hat{\beta}_j)} \to 0$$
\vspace{-3pt}
\end{formulabox}

\textbf{Resultado:} Se acepta $H_0: \beta_j = 0$ incorrectamente.

% ========================================
% COLUMNA 3
% ========================================

\section{III. DIAGNÓSTICO COMPLETO}
\sectionline

\subsection{SÍNTOMAS OBSERVABLES}

\textbf{1. Paradoja $R^2$ alto - $t$ bajos}
\begin{itemize}
    \item $R^2 > 0.90$ (modelo global significativo)
    \item $F$ significativo
    \item $t$ individuales NO significativos
\end{itemize}

\textbf{2. Sensibilidad extrema}
\begin{itemize}
    \item Coeficientes cambian drásticamente
    \item Signos incorrectos económicamente
    \item Inestabilidad al agregar/quitar datos
\end{itemize}

\textbf{3. Intervalos amplios}
\begin{itemize}
    \item IC muy anchos
    \item Errores estándar grandes
\end{itemize}

\subsection{CÁLCULO DEL FIV}

\textbf{Procedimiento paso a paso:}

\textbf{Paso 1:} Para cada $X_j$, regresar:
\begin{formulabox}
\vspace{-3pt}
$$X_j = \delta_0 + \sum_{i \neq j} \delta_i X_i + v_j$$
\vspace{-3pt}
\end{formulabox}

\textbf{Paso 2:} Obtener $R_j^2$ de esta regresión

\textbf{Paso 3:} Calcular:
$$\FIV_j = \frac{1}{1-R_j^2}$$

\textbf{Paso 4:} Interpretar según criterios

\subsection{CRITERIOS DE DECISIÓN FIV}

\begin{center}
{\scriptsize
\begin{tabular}{@{}ll@{}}
\toprule
\textbf{Valor} & \textbf{Acción} \\
\midrule
$\FIV < 5$ & Ninguna acción \\
$5 \leq \FIV < 10$ & Monitorear \\
\rowcolor{WarningBox}
$\FIV \geq 10$ & \textbf{ACCIÓN URGENTE} \\
$\FIV > 100$ & Rediseñar modelo \\
\bottomrule
\end{tabular}
}
\end{center}

\subsection{TEST DE FARRAR-GLAUBER}

\textbf{Etapa 1: Test $\chi^2$ (Global)}

\begin{formulabox}
\vspace{-3pt}
$$\chi^2_{calc} = -\left[n - 1 - \frac{1}{6}(2k+5)\right] \ln|R_{XX}|$$
\vspace{-3pt}
\end{formulabox}

\textbf{Distribución:} $\chi^2_{calc} \sim \chi^2_{[k(k-1)/2]}$

\textbf{Decisión:} Rechazar $H_0$ si $\chi^2_{calc} > \chi^2_{\alpha,gl}$

\textbf{Etapa 2: Test $F$ (Localización)}

Para cada variable:
$$F_j = \frac{R_j^2/(k-2)}{(1-R_j^2)/(n-k+1)} \sim F_{(k-2,n-k+1)}$$

\textbf{Etapa 3: Test $t$ (Pares)}

$$t_{ij} = r_{ij \cdot resto} \sqrt{\frac{n-k}{1-r_{ij \cdot resto}^2}}$$

% ========================================
% COLUMNA 4
% ========================================

\subsection{ÍNDICE DE CONDICIÓN}

\textbf{Número de condición:}
\begin{formulabox}
\vspace{-3pt}
$$\kappa = \frac{\lambda_{max}}{\lambda_{min}}$$
\vspace{-3pt}
\end{formulabox}

\textbf{Índice de condición:}
\begin{formulabox}
\vspace{-3pt}
$$CI = \sqrt{\kappa} = \sqrt{\frac{\lambda_{max}}{\lambda_{min}}}$$
\vspace{-3pt}
\end{formulabox}

\textbf{Criterios (Belsley, Kuh, Welsch):}
\begin{itemize}
    \item $CI < 10$: Sin problema
    \item $10 \leq CI < 30$: Moderada a fuerte
    \item $CI \geq 30$: \textcolor{red}{\textbf{Severa}}
\end{itemize}

\subsection{REGLA DE KLEIN}

Multicolinealidad problemática si:
\begin{formulabox}
\vspace{-3pt}
$$R_j^2 > R^2_{modelo}$$
\vspace{-3pt}
\end{formulabox}

\subsection{CORRELACIONES SIMPLES}

\textbf{Regla práctica:}
$$|r_{ij}| > 0.80 \implies \text{Bandera roja}$$

\begin{warningbox}
\textbf{ADVERTENCIA:} Correlaciones bajas NO garantizan ausencia de multicolinealidad múltiple.
\end{warningbox}

\section{IV. REMEDIOS Y SOLUCIONES}
\sectionline

\subsection{FILOSOFÍA: ¿ACTUAR?}

\begin{center}
{\scriptsize
\begin{tabular}{@{}lll@{}}
\toprule
\textbf{Objetivo} & \textbf{¿Importa?} & \textbf{Acción} \\
\midrule
Predicción & NO & No hacer nada \\
Inferencia & SÍ & Remedios \\
Interpretación & SÍ & Remedios \\
\bottomrule
\end{tabular}
}
\end{center}

\begin{warningbox}
\textbf{RECORDAR:} Multicolinealidad NO sesga, solo aumenta varianza.
\end{warningbox}

\subsection{1. INFORMACIÓN A PRIORI}

\textbf{Ejemplo: Cobb-Douglas}

Si teoría dice $\beta_2 + \beta_3 = 1$:
\begin{formulabox}
\vspace{-3pt}
$$\ln(Q/L) = \beta_1 + \beta_2 \ln(K/L) + u$$
\vspace{-3pt}
\end{formulabox}

\textbf{Ventaja:} Elimina completamente el problema.

\subsection{2. PRIMERAS DIFERENCIAS}

Para series de tiempo:
\begin{formulabox}
\vspace{-3pt}
$$\Delta Y_t = \beta_2 \Delta X_{2t} + \beta_3 \Delta X_{3t} + \Delta u_t$$
\vspace{-3pt}
\end{formulabox}

\textbf{Elimina:} Tendencias comunes

\textbf{Costo:} Pérdida de info de niveles

\subsection{3. RAZONES/COCIENTES}

En lugar de $Q = \beta_2 K + \beta_3 L$:
$$\frac{Q}{L} = \beta_2 \frac{K}{L} + \beta_3$$

\subsection{4. ELIMINAR VARIABLES}

\textbf{Criterio:} Identificar variable con $\FIV_{max}$ y eliminar

\begin{warningbox}
\textbf{RIESGO:} Sesgo de especificación si variable es relevante
\end{warningbox}

\subsection{5. MÁS DATOS}

\begin{formulabox}
\vspace{-3pt}
$$\Var(\hat{\beta}_j) \propto \frac{1}{n}$$
\vspace{-3pt}
\end{formulabox}

Para reducir $\text{se}$ a la mitad: $n_{nuevo} = 4n_{original}$

\subsection{6. REGRESIÓN RIDGE}

\textbf{Estimador Ridge:}
\begin{formulabox}
\vspace{-3pt}
$$\betahat_R = (\Xmat'\Xmat + \lambda I)^{-1}\Xmat'\Ymat$$
\vspace{-3pt}
\end{formulabox}

\textbf{Trade-off:}
\begin{itemize}
    \item \textcolor{red}{Introduce sesgo}
    \item \textcolor{OliveGreen}{Reduce varianza}
    \item $\lambda > 0$ controla balance
\end{itemize}

\subsection{7. LASSO}

Minimiza:
$$\sum (Y_i - \Xmat_i'\boldsymbol{\beta})^2 + \lambda \sum |\beta_j|$$

\textbf{Ventaja:} Selección automática de variables (coefs = 0)

\subsection{8. PCA}

\textbf{Componentes principales:}
$$PC_j = \sum_{i=1}^k a_{ij} X_i$$

\textbf{Ventaja:} Ortogonalidad perfecta

\textbf{Desventaja:} \textcolor{red}{Pérdida de interpretabilidad}

\section{V. APLICACIONES}
\sectionline

\subsection{MODELO INFLACIÓN MONETARISTA}

\begin{formulabox}
\vspace{-3pt}
$$\dot{P}_t = \alpha + \sum_{i=0}^n m_i \dot{M}_{t-i} + u_t$$
\vspace{-3pt}
\end{formulabox}

\textbf{Problema:} $\text{Corr}(\dot{M}_t, \dot{M}_{t-1}) \approx 0.85$

\textbf{Solución:} Rezagos de Almon
$$m_i = \gamma_0 + \gamma_1 i + \gamma_2 i^2$$

\textbf{Resultado empírico:} $\sum m_i \approx 1.03$ (neutralidad)

\subsection{FUNCIÓN COBB-DOUGLAS}

\textbf{Original:}
$$\ln Q = \beta_1 + \beta_2 \ln K + \beta_3 \ln L + u$$

\textbf{Problema:} $\text{Corr}(\ln K, \ln L) > 0.90$

\textbf{Remedios:}
\begin{enumerate}
    \item Restricción: $\ln(Q/L) = \beta_1 + \beta_2 \ln(K/L) + u$
    \item Diferencias: $\Delta \ln Q_t = \beta_2 \Delta \ln K_t + \beta_3 \Delta \ln L_t$
\end{enumerate}

\section{VI. CÓDIGO PRÁCTICO}
\sectionline

\subsection{R (TIDYVERSE)}

{\scriptsize\ttfamily
\textbf{Calcular FIV:}\\
library(car)\\
vif(modelo)\\[3pt]
\textbf{Tolerancia:}\\
1/vif(modelo)\\[3pt]
\textbf{Ridge:}\\
library(glmnet)\\
ridge <- glmnet(x, y, alpha=0)\\[3pt]
\textbf{LASSO:}\\
lasso <- glmnet(x, y, alpha=1)
}

\subsection{PYTHON (STATSMODELS)}

{\scriptsize\ttfamily
\textbf{FIV:}\\
from statsmodels.stats\\
.outliers\_influence import\\
variance\_inflation\_factor\\[3pt]
vif = [variance\_inflation\_factor\\
(X.values, i) for i in range(X.shape[1])]\\[3pt]
\textbf{Ridge/LASSO:}\\
from sklearn.linear\_model\\
import Ridge, Lasso
}

\subsection{STATA}

{\scriptsize\ttfamily
\textbf{VIF:}\\
estat vif\\[3pt]
\textbf{Restricciones:}\\
constraint define 1 beta2+beta3=1\\
cnsreg y x1 x2, constraints(1)
}

\section{VII. REFERENCIA RÁPIDA}
\sectionline

\subsection{FÓRMULAS ESENCIALES}

\begin{center}
{\tiny
\begin{tabular}{@{}ll@{}}
\toprule
\textbf{Concepto} & \textbf{Fórmula} \\
\midrule
MCO & $\betahat = (\Xmat'\Xmat)^{-1}\Xmat'\Ymat$ \\
Var-Cov & $\Var(\betahat) = \sigma^2(\Xmat'\Xmat)^{-1}$ \\
FIV & $\FIV_j = 1/(1-R_j^2)$ \\
TOL & $\TOL_j = 1-R_j^2$ \\
F-G & $\chi^2 = -[n-1-\frac{1}{6}(2k+5)]\ln|R_{XX}|$ \\
CI & $CI = \sqrt{\lambda_{max}/\lambda_{min}}$ \\
Ridge & $\betahat_R = (\Xmat'\Xmat + \lambda I)^{-1}\Xmat'\Ymat$ \\
\bottomrule
\end{tabular}
}
\end{center}

\subsection{CHECKLIST DIAGNÓSTICO}

\begin{enumerate}
    \item[$\square$] Calcular matriz correlación
    \item[$\square$] Verificar $|r_{ij}| > 0.80$
    \item[$\square$] Calcular FIV para cada variable
    \item[$\square$] Identificar $\FIV > 10$
    \item[$\square$] Test Farrar-Glauber (si necesario)
    \item[$\square$] Análisis de valores propios (CI)
    \item[$\square$] Regresiones auxiliares
    \item[$\square$] Decidir remedio apropiado
\end{enumerate}

\subsection{DECISIONES CLAVE}

\textbf{Si $\FIV > 10$:}
\begin{enumerate}
    \item ¿Objetivo predicción? → No hacer nada
    \item ¿Teoría sugiere restricción? → Info a priori
    \item ¿Series de tiempo? → Diferenciación
    \item ¿Variable redundante? → Eliminar
    \item ¿Ninguno aplica? → Ridge/LASSO
\end{enumerate}


\section{VIII. CASOS ESPECIALES}
\sectionline

\subsection{TRAMPA VARIABLE DUMMY}

\textbf{Error común:}
$$Y = \beta_0 + \beta_1 D_1 + \beta_2 D_2 + u$$
donde $D_1 + D_2 = 1$ (intercepto)

\textbf{Solución:} Omitir una categoría

\subsection{TÉRMINOS POLINOMIALES}

Para $Y = \beta_1 + \beta_2 X + \beta_3 X^2 + u$:

\textbf{Centrar:} $x^* = X - \bar{X}$
$$Y = \beta_1^* + \beta_2^* x^* + \beta_3^* (x^*)^2 + u$$

\subsection{VARIABLES DE ESCALA}

Si $X_2 = c \cdot X_3$ (ej: kg vs. libras):

\textbf{Colinealidad perfecta} → Usar solo una

\section{IX. INTERPRETACIÓN}
\sectionline

\subsection{NATURALEZA DEL PROBLEMA}

\begin{warningbox}
La multicolinealidad es problema de:
\begin{itemize}
    \item \textbf{GRADO} (no existencia)
    \item \textbf{MUESTRA} (no población)
    \item \textbf{PRECISIÓN} (no sesgo)
\end{itemize}
\end{warningbox}

\subsection{NO AFECTA}

\begin{itemize}
    \item Insesgadez: $\E(\betahat) = \boldsymbol{\beta}$
    \item Consistencia
    \item Propiedad BLUE
    \item Predicción (si estructura persiste)
    \item $R^2$ global
\end{itemize}

\subsection{SÍ AFECTA}

\begin{itemize}
    \item Varianzas individuales ↑
    \item Errores estándar ↑
    \item Intervalos de confianza ↑
    \item Poder de pruebas $t$ ↓
    \item Estabilidad numérica ↓
\end{itemize}

\subsection{CONTEXTO ECONÓMICO}

\textbf{Pregunta clave:} ¿Qué necesito?

\begin{center}
{\scriptsize
\begin{tabular}{@{}ll@{}}
\toprule
\textbf{Si necesito...} & \textbf{Entonces...} \\
\midrule
Predicción $\hat{Y}$ & Tolerar colinealidad \\
Inferencia sobre $\beta_j$ & Aplicar remedios \\
Interpretación causal & Remedios + teoría \\
Elasticidades precisas & Remedios esenciales \\
\bottomrule
\end{tabular}
}
\end{center}

\section{X. NOTAS FINALES}
\sectionline

\subsection{MITOS COMUNES}

\begin{enumerate}
    \item \textcolor{red}{FALSO:} ``Alta correlación simple = multicolinealidad''
    
    \textcolor{OliveGreen}{VERDAD:} Puede haber multicolinealidad con $r_{ij}$ bajos
    
    \item \textcolor{red}{FALSO:} ``Multicolinealidad sesga estimadores''
    
    \textcolor{OliveGreen}{VERDAD:} Solo aumenta varianza, no sesgo
    
    \item \textcolor{red}{FALSO:} ``Siempre hay que corregirla''
    
    \textcolor{OliveGreen}{VERDAD:} Depende del objetivo
\end{enumerate}

\subsection{BIBLIOGRAFÍA ESENCIAL}

\begin{itemize}
    \item \textbf{Gujarati \& Porter} (2009): Cap. 10
    \item \textbf{Wooldridge} (2015): Cap. 3-4
    \item \textbf{Greene} (2018): Sec. 4.6
    \item \textbf{Belsley et al.} (1980): Texto completo
\end{itemize}

\subsection{VALORES CRÍTICOS}

\textbf{$\chi^2$ (Farrar-Glauber, $\alpha=0.05$):}

{\tiny
\begin{tabular}{@{}cc@{}}
gl=3: 7.815 & gl=6: 12.592 \\
gl=10: 18.307 & gl=15: 24.996 \\
\end{tabular}
}

\vspace{6pt}

\begin{center}
{\scriptsize\sffamily
\textcolor{gray}{---}\\
\textbf{UPTC Tunja} | Econometría Avanzada\\
Emanuel | Febrero 2026\\
\texttt{github.com/uptc-econometrics}
}
\end{center}

\end{multicols*}

\end{document}
